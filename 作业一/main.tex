%! Tex program = xelatex   
\documentclass{homework}
\usepackage{xeCJK}
\usepackage{amsmath}
\usepackage{booktabs} %表格
\usepackage{longtable}
% \setmainfont{Times New Roman}
\setCJKmainfont{Kaiti SC}
% \setCJKfamilyfont{song}{Songti SC}
\renewcommand{\baselinestretch}{1.3} %行间距
\author{朱浩泽 1911530}
\class{软件工程作业一}
\date{\today}
\title{\Large{关于代码托管平台的调查}}

\graphicspath{{./media/}}

\begin{document} \maketitle

\question \large{作为一个计算机大类的学生,你通常使用的代码托管平台是什么?}

\normalsize{答:我常用的代码托管平台是微软公司的github.}

\question \large{比较代码托管平台,列出优缺点。}
\normalsize

答:为了得出较好的平台比较效果,我们选择市面上主流的代码托管平台github(最主流且本人一直使用)、gitee(较为主流但本人没用过)、gitlab(不是很主流但本人曾经使用过)进行比较。
\begin{itemize}
	\item 关于网络环境
		\begin{itemize}
			\item[$\circ$] \textbf{Github}  网络环境不是很稳定,国内用户经常需要利用科学上网的方式才能链接登陆或上传、拉取仓库。
			\item[$\circ$] \textbf{Gitlab}  网络环境在可以使用是较好,但有时服务器会宕机维护,这段时间内通过任何方法都无法登陆或上传拉取自己的仓库。
			\item[$\circ$] \textbf{Gitee}  目前国内由深圳市奥思网络科技有限公司版权所有提供服务, 网络问题相对较少。
		\end{itemize}
	\item 关于联合生态 
		\begin{itemize}
			\item[$\circ$] \textbf{Github}  与许多集成开发环境有着紧密联系,比如vs code、overleaf等众多集成终端,与此同时世界上许多的主流程序员社区也都支持用github登陆,其生态环境是另外两个代码托管平台不能匹敌的。
			\item[$\circ$] \textbf{Gitlab}  GitLab被IBM,Sony,JülichResearchCenter,NASA,Alibaba,Invincea,O’ReillyMedia,Leibniz-Rechenzentrum(LRZ),CERN,SpaceX等组织使用。但面向大众的合作的终端或社区相对较少,生态环境相对较为闭塞。
			\item[$\circ$] \textbf{Gitee}   与国内一些主流产品进行合作,如提供微信小程序等,同时也提供主流浏览器的插件(但优化一般,更新不及时)以及一些集成编译器的插件,总体来说生态环境在国内市场较好,但放眼世界相对github。 
		\end{itemize} 
	\item 关于功能
		\begin{table}[!htbp]
			\centering
			\begin{tabular}{ccccccccccc}
			\toprule  
			& \textbf{GitHub}& \textbf{Gitee}& \textbf{Gitlab}\\
				\midrule
				代码托管,支持 Git/SVN& 仅支持git & 均支持& 仅支持git	\\	
				开源项目、代码片段& 支持 &支持&支持		\\
				Issue& 	支持 &支持&支持		\\
				Wiki& 	支持 &支持&支持		\\
				Fork + Pull Request	& 支持 &支持&支持		\\
				组织& 	支持 &支持&支持		\\
				在线 IDE& 不支持& 支持& 支持\\
				\bottomrule
			\end{tabular}
			\caption{功能表}
		\end{table}
		\begin{table}[!htbp]
			\centering
			\begin{tabular}{ccccccccccc}
			\toprule  
			 & \textbf{GitHub}& \textbf{Gitee}& \textbf{Gitlab}\\
				\midrule
				仓库自动备份& 	不支持& 支持& 不支持&\\
				禁止 Git 强推& 	不支持&  支持& 不支持&\\
				支持仓库访问 IP 限制& 不支持& 支持& 不支持&\\
				企业级研发协作& 收费 &五人免费& 不支持&\\
				敏捷开发管理& 不支持&	支持&	不支持&\\
				任务看板(可灵活定义)& 不支持& 支持& 不支持&\\		
				支持多级任务、关联任务& 不支持& 支持& 不支持&\\		
				自动代码质量分析& 不支持& 支持&	不支持&\\	
				快捷生成工作周报& 不支持& 支持&	不支持&\\	
				代码克隆检测& 不支持& 支持&	不支持&\\
				自动生成 JavaDoc/PHPDoc	& 不支持& 支持&	不支持&\\
				多语言 README 自动渲染& 不支持&	 支持& 不支持&\\	
				支持微信/钉钉通知& 	不支持&	支持& 不支持&\\
				完善的管理界面& 不支持& 不支持& 支持\\
				完善的权限控制& 局限& 局限& 完善\\
				订阅& 支持& 支持& 支持\\
				文本渲染&支持&支持&支持\\
				讨论组&支持&支持&不支持\\
				协作图谱&支持&不支持&不支持\\
				Gist&支持&不支持&不支持\\
				\bottomrule
			\end{tabular}
			\caption{功能表(续)}
		\end{table}
		\\
		\par 可以看出,gitee推出了一些新功能。这些功能大多数针对公司的项目开发环境,并且与中国市场上的主流软件和工作环境紧密结合,可以在一定程度上提高工作效率。但这三个代码托管平台在基本功能上并无本质区别,其提供的服务基本可以保证日常使用需求。
	\item 关于语言
		\begin{itemize}
			\item[$\circ$] \textbf{Github} 仅有英文版本,对中国用户不够友好
			\item[$\circ$] \textbf{Gitlab} 支持中文
			\item[$\circ$] \textbf{Gitee} 支持中文 
		\end{itemize}
	\item 使用场景
		\begin{itemize}
			\item[$\circ$] \textbf{Github} 世界上最大的开源代码托管平台,汇聚了世界上大多数的开源顶尖项目,可以帮助程序员之间互相交流和学习,也可以参与各式各样的开源项目以及托管自己的代码仓库。
			\item[$\circ$] \textbf{Gitlab} 基于Git实现的在线代码仓库软件,可以用GitLab搭建一个类似于GitHub一样的仓库,但是GitLab有完善的管理界面和权限控制,具有较强的可塑性和自定义性,一般用于在企业、学校等内部网络搭建Git私服。
			\item[$\circ$] \textbf{Gitee} 全中文环境,而且大部分用户都是国人,不会出现不稳定情况,但优秀的库相对于GitHub也少。提供免费的 Git 仓库,还集成了代码质量检测、项目演示等功能。对于团队协作开发,Gitee 还提供了项目管理、代码托管、文档管理的服务,5人以下小团队免费等许多面向中国特色的功能,适合中国公司团队做开发项目使用。

		\end{itemize}
\end{itemize}


\question \large{开发一个代码托管平台应如何改进?}

\normalsize 答:开发一个代码托管平台,我认为第一步的要做的事情是去调研了解用户需求。

\end{document}