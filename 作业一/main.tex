%! Tex program = xelatex   
\documentclass{homework}
\usepackage{xeCJK}
\usepackage{amsmath}
\usepackage{booktabs} %表格
% \setmainfont{Times New Roman}
\setCJKmainfont{Kaiti SC}
% \setCJKfamilyfont{song}{Songti SC}
\renewcommand{\baselinestretch}{1.3} %行间距
\author{朱浩泽 1911530}
\class{软件工程作业一}
\date{\today}
\title{\Large{关于代码托管平台的调查}}

\graphicspath{{./media/}}

\begin{document} \maketitle

\question \large{作为一个计算机大类的学生,你通常使用的代码托管平台是什么?}

\normalsize{答:我常用的代码托管平台是微软公司的github.}

\question \large{比较代码托管平台,列出优缺点。}
\normalsize

答:为了得出较好的平台比较效果,我们选择市面上主流的代码托管平台github(最主流且本人一直使用)、gitee(较为主流但本人没用过)、gitlab(不是很主流但本人曾经使用过)进行比较。
\begin{itemize}
	\item 关于网络环境
		\begin{itemize}
			\item[$\circ$] \textbf{Github}  网络环境不是很稳定,国内用户经常需要利用科学上网的方式才能链接登陆或上传、拉取仓库
			\item[$\circ$] \textbf{Gitlab}  网络环境在可以使用是较好,但有时服务器会宕机维护,这段时间内通过任何方法都无法登陆或上传拉取自己的仓库
			\item[$\circ$] \textbf{Gitee}  目前国内由阿里巴巴提供服务, 网络问题相对较少
		\end{itemize}
	\item 
\end{itemize}

\question 开发一个代码托管平台应如何改进?

答:开发一个代码托管平台,我认为第一步的要做的事情是去调研了解用户需求。






\end{document}